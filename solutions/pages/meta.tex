% meta.tex
%
% Aetf <aetf@unlimitedcodeworks.xyz>
% Copyright 2016 Aetf <aetf@unlimitedcodeworks.xyz>
%
% multiple1902 <multiple1902@gmail.com>
% Copyright 2011~2012, multiple1902 (Weisi Dai)
%
% Project Home: https://github.com/Aetf/xjtuthesis
%
% It is strongly recommended that you read documentations located at
%   https://github.com/Aetf/xjtuthesis/wiki
% in advance of your compilation if you have not read them before.
%
% This work may be distributed and/or modified under the
% conditions of the LaTeX Project Public License, either version 1.3
% of this license or (at your option) any later version.
% The latest version of this license is in
%   http://www.latex-project.org/lppl.txt
% and version 1.3 or later is part of all distributions of LaTeX
% version 2005/12/01 or later.
%
% This work has the LPPL maintenance status `maintained'.
%
% The Current Maintainer of this work is Aetf.
%

% 标题,中文
\ctitle{这是一个标题}

% 作者,中文
\cauthor{张三}

% 学科,中文,本科生不需要
\csubject{}

% 学院,专业,班级,中文,本科毕业论文需要
\ccollege{能源与动力工程}
\cmajor{能源与动力工程A}
\cclass{能动A71}

% 设计所在单位,中文,本科毕业论文需要
\corgnization{西安交通大学}

% 学号
\stunum{1234567890}

% 导师姓名,中文
\csupervisor{李四}

% 关键词,中文。用全角分号「;」分割
% 研究生的应首先从《汉语主题词表》中摘选
\ckeywords{}

% 提交日期
\cproddate{{\color{black}\the\year} 年\ {\color{black}\the\month} 月}

% 论文类型,中文,本科生不需要
% 从理论研究、应用基础、应用研究、研究报告、软件开发、设计报告、案例分析、调研报告、其它中选择
\ctype{}

% 论文标题,英文
\etitle{An Awesome Paper Title}

% 作者姓名,英文
\eauthor{Author Name}

% 学科,英文,本科生不需要
\esubject{}

% 导师姓名,英文
\esupervisor{}

% 关键词,英文。用半角分号和一个半角空格「; 」分割
\ekeywords{Wikipedia; Free encyclopedia; Winner; Good morning}

% 学科门类,英文
% 从Philosophy(哲学)、Economics(经济学)、Law(法学)、Education(教育学)、Arts(文学)、
%   Science(理学)、Engineering Science(工学)、Medicine(医学)、Management Science(管理学)中选择 
\ecate{}

% 提交日期,英文,本科生不需要
% 应当和 cproddate 保持一致
\eproddate{\monthname{\month}\ \the\year}

% 论文类型,英文,本科生不需要
% 从Theoretical Research(理论研究)、Application Fundamentals(应用基础)、Applied Research(应用研究)、
%   Research Report(研究报告)、Software Development(软件开发)、Design Report(设计报告)、
%   Case Study(案例分析)、Investigation Report(调研报告)、其它(Other)中选择
\etype{}

% 摘要,中文。段间空行
\cabstract{
}

% 摘要,英文。段间空行
\eabstract{
}
